\documentclass[11pt]{article}

\usepackage{amsmath,amssymb,amsfonts}
\usepackage{graphicx}
%\usepackage{feynman}
\usepackage{bm}
\usepackage{pythonhighlight}

\setlength{\topmargin}{-.5in} \setlength{\textheight}{9.25in}
\setlength{\oddsidemargin}{0in} \setlength{\textwidth}{6.8in}
\begin{document}
\Large
\noindent\textbf{CMS Office Hours} \\
\noindent\textbf{Project 1: Create an Integration Code \hfill 2020-Aug-10}
\medskip\hrule
%%%%%%%%%%%%%%%%%%
\vspace{5mm}

Use Python to develop a script which integrates a User-defined 1-dimensional function 
$\left( f(x) \right)$
over User-specified integration bounds. 
% or class which returns the \textit{value of integration}
% when the User provides a 1-dimensional function ($f(x)$) which is to be integrated
% over the integration bounds. 

\textbf{Example:}
\begin{python}
user_defined_function = 2 * sin(x)
your_integrator(user_defined_function, 0, pi)
# returns: 4.0
\end{python}
% \pyth{print("Hello World!")}  # Inline python syntax in LaTeX. 

Here is a general template to structure your code (or use your own):

\begin{python}
"""
This code returns the value of integration when the User 
specifies a function and the integration bounds. 

Example:
    user_defined_function = 2 * sin(x)
    your_integrator(user_defined_function, 0, np.pi)  # (fn, x_min, x_max)
    # returns: 4.0
"""
import numpy as np

#--- User-defined/Global variables. 
x_min = 0.0
x_max = 2.0
def fn(x):
    return 2 * np.sin(x)

#--- Script functions.
def integrate(fn, x_min, x_max):
    """Take in a function (fn), x_min, x_max and return the value of integration."""
    # Your code goes here.

def calc_area_of_rect(width, height):
    """Calculate the area of a rectangle."""
    return width * height

#--- You will probably need more functions than just the two above.       

if __name__ == "__main__":
    # When you do: `python this_script.py', then this code will be executed.
    integrate(fn, x_min, x_max)
\end{python}

\end{document} 